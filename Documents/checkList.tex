\chapter{چک لیست  آزمون}

\section{بررسی کلاس‌ها}
\begin{enumerate}
\item کلاس Project

\begin{description}
\item[$\bullet$ Completeness]  این کلاس مسئولیت رفتارهای مخصوص یک پروژه را دارد و با ارائه عملیاتی نظیر addTechnology، getTechnologies، getProjectManager، setProjectManager خواسته‌های مشتری را برآورده می‌کند.
\par
\item[$\bullet$ Sufficiency] برای رسیدن به این ویژگی، خاصیت‌های اضافه از این کلاس جدا شده و به کلاس‌های دیگر سپرده شده است نظیر مدیریت مجموعه پروژه‌ها در کلاس ProjectCatalogue، افزودن منبع و تخصیص منبع به پروژه در کلاس ProjectRequirementCatalogue.
\par
\item[$\bullet$ Primitiveness] تمام رفتارهای این کلاس در راستای مدیریت یک پروژه است و هیچ کلاس دیگری این کار را انجام نمی‌دهد و هر رفتار یک کار ساده انجام می‌دهد نظیر دادن و گرفتن تعداد کاربران، زمان شروع و پایان پروژه.
\end{description}

\item کلاس Unit
\begin{description}
\item[$\bullet$ Completeness]  این کلاس مسئولیت رفتارهای مخصوص یک واحد را دارد و با ارائه عملیاتی نظیر getAvailableResources، getRequirements، addRequirement خواسته‌های مشتری را برآورده می‌کند.
\par
\item[$\bullet$ Sufficiency] برای رسیدن به این ویژگی، خاصیت‌های اضافه از این کلاس جدا شده و به کلاس‌های دیگر سپرده شده است نظیر مدیریت مجموعه واحدها در کلاس UnitCatalogue.
\par
\item[$\bullet$ Primitiveness] تمام رفتارهای این کلاس در راستای مدیریت یک واحد است و هیچ کلاس دیگری این کار را انجام نمی‌دهد و هر رفتار یک کار ساده انجام می‌دهد نظیر مقدار دهی و گرفتن نام واحد، منابع و نیازمندی‌های واحد.
\end{description}
\item کلاس System

\begin{description}
\item[$\bullet$ Completeness]  این کلاس مسئولیت رفتارهای مخصوص یک سیستم از پروژه را دارد و با ارائه عملیاتی نظیر addModule خواسته‌های مشتری را برآورده می‌کند.
\par
\item[$\bullet$ Sufficiency] برای رسیدن به این ویژگی، خاصیت‌های اضافه از این کلاس جدا شده و به کلاس‌های دیگر سپرده شده است نظیر مدیریت مجموعه سیستم‌ها در کلاس Project.
\par
\item[$\bullet$ Primitiveness] تمام رفتارهای این کلاس در راستای مدیریت یک سیستم است و هیچ کلاس دیگری این کار را انجام نمی‌دهد و هر رفتار یک کار ساده انجام می‌دهد نظیر اضافه کردن و گرفتن ماژول و گرفتن نام سیستم. 
\end{description}

\item کلاس Module

\begin{description}
\item[$\bullet$ Completeness]  این کلاس مسئولیت رفتارهای مخصوص یک ماژول را دارد و با ارائه عملیاتی نظیر addModuleModification، getModuleModifications، addDeveloper خواسته‌های مشتری را برآورده می‌کند.
\par
\item[$\bullet$ Sufficiency] برای رسیدن به این ویژگی، خاصیت‌های اضافه از این کلاس جدا شده و به کلاس‌های دیگر سپرده شده است نظیر مدیریت مجموعه ماژول‌ها در کلاسSystem  و اضافه کردن و گرفتن اطلاعات مربوط به تغییرات اعمال شده برروی ماژول در کلاس ModuleModification.
\par
\item[$\bullet$ Primitiveness] تمام رفتارهای این کلاس در راستای مدیریت یک ماژول است و هیچ کلاس دیگری این کار را انجام نمی‌دهد و هر رفتار یک کار ساده انجام می‌دهد نظیر مقداردهی و گرفتن زمان شروع و پایان توسعه ماژول و نام آن. 
\end{description}

\item کلاس Resource

\begin{description}
\item[$\bullet$ Completeness] این کلاس مسئولیت رفتارهای مشترک بین انواع منابع را برعهده دارد.
\par
\item[$\bullet$ Sufficiency] برای رسیدن به این ویژگی، خاصیت‌های اضافه‌ از این کلاس جدا شده و خاصیت‌های مخصوص هر نوع منبع، به کلاس مخصوص آن اضافه شده است نظیر مدیریت مجموعه منابع در ResourceCatalogue و خاصیت‌ها و رفتارهای مربوط به منابع انسانی در کلاس HumanResource و خاصیت‌ها و رفتارهای مربوط به منابع فیزیکی در کلاس PhysicalResource و خاصیت‌ها و رفتارهای مربوط به منابع مالی در کلاس MonetaryResource و خاصیت‌ها و رفتارهای مربوط به منابع اطلاعاتی در InformationResource.
\par
\item[$\bullet$ Primitiveness] تمام رفتارهای این کلاس در راستای مدیریت یک منبع است و هیچ کلاس دیگری این کار را انجام نمی‌دهد و هر رفتار یک کار ساده انجام می‌دهد نظیر مقدار دهی و گرفتن حالت منبع و وضعیت در دسترس بودن آن. 
\end{description}
\end{enumerate}

موارد دیگری که طبق چک‌لیست آزمون در نمودار کلاس مورد بررسی قرار گرفته‌اند، عبارتند از:
\begin{itemize}
	\item افزودن navigability به روابط
	\item افزودن multiplicity به هر دو طرف روابط
	\item افزودن نام نقش به مقصد رابطه
	\item پیاده‌سازی روابط یک‌به‌یک، یک‌به‌چند، چند‌به‌یک و چندبه‌چند
	\item تبدیل کلاس‌های association به وضعیت قابلیت پیاده‌سازی یا به عبارت دیگر reify شده
\end{itemize}
