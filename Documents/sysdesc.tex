\chapter{شرح سیستم برنامه‌ریزی منابع سازمانی}
سیستم برنامه‌ریزی برای منابع سازمانی، یکی از انواع سیستم‌های کسب و کار است که مدیریت منابع سازمانی شامل منابع انسانی، منابع اطلاعاتی، منابع مالی (نقدی و غیرنقدی) و منابع فیزیکی را تسهیل می‌کند. سیستم برنامه‌ریزی پیش رو برای استفاده در سازمانی ایجاد شده است که در حوزه‌ی ایجاد سیستم‌های نرم‌افزاری فعالیت می‌کند. \par
در این سازمان هر پروژه شامل تعدادی سیستم و هر سیستم خود شامل تعدادی ماژول است. در هر پروژه تیمی متشکّل از افرادی از واحدهای چهارگانه‌ی مهندسی نیازمندی‌ها، تحلیل، طرّاحی، و پیاده‌سازی و نگه‌داری سازمان فعالیّت می‌کنند. همچنین هر فرد در این سازمان با یکی از مراتب شغلی کارمند معمولی، مدیر میانی و مدیر سطح بالا فعالیت می‌کند. سطح دسترسی و میزان تعامل هر فرد با سیستم برنامه‌ریزی در هر پروژه به طور جداگانه و توسط فردی با مرتبه‌ی شغلی بالاتر تعیین می‌گردد. \par
سیستم برنامه‌ریزی قابلیت ثبت فرایند ایجاد و نگه‌داری را پشتیبانی می‌کند که امکان ثبت اعضا، زمان صرف شده، و منابع استفاده شده در روند ثبت و تغییر ماژول‌ها را در اختیار کاربران می‌گذارد. این سیستم همچنین امکان ثبت منابع موجود، ثبت نیازمندی‌های کنونی و ثبت اندازه‌ی پروژه‌های نرم‌افزاری را فراهم می‌کند. به وسیله‌ی همین اطلاعات است که امکان گزارش‌گیری از سیستم برنامه‌ریزی با هدف دریافت گزارش منابع موجود و مورد نیاز و دریافت گزارش جریان چرخشی مصرف فراهم می‌شود. همچنین این سیستم با کمک همین اطلاعات در پیش‌بینی برای تخمین منابع و یافتن نیازمندی‌های ضروری به کاربران کمک می‌کند. \par

