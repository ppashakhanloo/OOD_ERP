
\chapter{مقدمات}
در این فصل، هر کدام از مستندات به طور خلاصه به همراه نقش آن‌ها آورده شده است.

\section{شرح مستندات}
\subsection{سند ریسک‌ها و نیازمندی‌ها}
به منظور کنترل عواقب ریسک‌ها و اطمینان از برآورده کردن نیازمندی‌های الزامی از دید مشتری، در این فصول علاوه بر عنوان و دسته‌بندی کردن ریسک‌ها و نیازمندی‌های موجود، به اولویت‌بندی آن‌ها پرداخته شده است. همچنین در مورد ریسک‌ها، سعی شده است راه حلی ارائه شود تا از خسارت‌های احتمالی هر یک تا حد مکان کاسته شود.

\subsection{سند موارد کاربرد}
این فصل، شامل سه بخش توصیف کنش‌گرها، نمودار های موارد کاربرد و توصیف آن‌ها است. به علت محوریت موارد کاربرد \LTRfootnote{Usecase-driven}در متدلوژی  فرایند یکپارچه \LTRfootnote{Unified Process}  این سند یکی از مهم‌ترین اسناد تولید شده است؛ چرا که یک ورودی ضروری در مراحل تحلیل، طراحی و آزمون به حساب می‌آید. موارد کاربرد به گونه‌ای هستند که می‌توانند فارغ از جزییات پیاده‌سازی، برای تعامل با مشتری استفاده شوند.

\subsection{کارت‌های CRC}
در این بخش، تلاش شده تا کلاس‌هایی که عمدتا در دامنه‌ی مساله قرار دارند به همراه وظایف و همکاران خود شناسایی شوند. از این کارت‌ها در ایجاد نمودارهای کلاس استفاده خواهد شد.

\subsection{نمودارهای فعالیت}
به ازای هر کدام از موارد کاربرد موجود در مستند، یک نمودار فعالیت رسم شده تا ترتیب اجرای گام‌های موارد کاربرد روشن شود. در گام بعدی، نمودارهای فعالیت دقیق‌تر خواهند شد. کاربرد اصلی این نمودار، در گذار از فضای وظیفه‌مندی مورد کاربرد به فضای شی‌گرای نمودار توالی است.

\subsection{نمونه‌ی واسط کاربری}
در این سند، تصاویری از نمونه‌ی واسط کاربری قرار داده شده است. این تصاویر برای ارتباط با مشتری و همچنین نمایش قابلیت‌های سیستم برنامه‌ریزی کاربرد دارد.

\subsection{نمودار کلاس‌های تحلیل}
در این سند، با کمک  کارت‌های CRC ، کلاس‌های حوزه‌ی مساله و روابط میان آن‌ها شناسایی و آورده شده است.


\subsection{نمودار بسته}
در این سند، زیرسیستم‌های سیستم برنامه‌ریزی در قالب پوشه‌هایی نمایش داده می‌شوند. در نمودار بسته، معماری سطح بالای سیستم قابل مشاهده است. هر بسته، شامل تعدادی کلاس است که دارای پیوستگی
\LTRfootnote{cohesion}
بالا هستند و بین بسته‌ها چسبندگی
\LTRfootnote{coupling}
پایین است.

\subsection{نمودارهای توالی تحلیل}
در این سند، به ازای هر نمودار فعالیت رسم شده، یک نمودار توالی رسم شده است. در نمودار توالی، موارد کاربرد با استفاده از operation ها و اشیایی که در کلاس‌های تحلیل شناسایی شده‌اند محقق می‌گردند.

\subsection{ نمودار مولفه}
نمودار مولفه چگونگی ارتباط کلاس ها و اشیاء سامانه را در طراحی فیزیکی سیستم نمایش می دهد. این مولفه ها شامل کلاس های طراحی هستند و نمود زمان اجرای آن ها artifact ها می باشند.
\subsection{نمودار کلاس‌های طراحی}
این سند شامل نمودار کلاس‌های طراحی است. این نمودار کامل شده کلاس‌های تحلیل است و ملاحظات قلمرو مسئله در آن در نظر گرفته شده است. در این نمودار، ساختار کلاس‌ها  و روابط بین آن‌ها قابل مشاهده است.
\subsection{نمودار توالی طراحی}
این سند شامل نمودارهای توالی است. در این نمودار با تاکید بر ترتیب انجام رویدادها، چگونگی تحقق موارد کاربرد  توسط نمونه‌هایی از کلاس‌های تحلیل مدل می‌شود. این نمودار نقش واسط را بین سایر نمودارها و کد شی گرا ایفا می کند و کد به صورت مستقیم از روی آن تولید می‌شود.

\subsection{شمای پایگاه داده}
این سند شامل شمای پایگاه داده است که به طور مستقیم برای طرّاحی و ایجاد جداول مورد نیاز در پیاده‌سازی استفاده شده است.

\subsection{واژه‌نامه}
در این پیوست، توضیحی راجع به واژگان به کار رفته در بقیه مستندات به همراه واژگان مترادف و متشابه آورده شده است. هدف این فصل، به وجود آوردن یک حوزه‌ی معنایی مشترک در مورد دامنه‌ی مساله میان مشتریان و اعضای تیم ایجاد است  تا جلوی سوء‌تفاهم‌های احتمالی در دامنه‌ی مساله تا حد امکان گرفته شود.

\subsection{الگوهای طراحی اعمال شده}
در این پیوست به توضیح الگوهای طرّاحی استفاده شده و موضوعیت استفاه از هر یک در برابر دیگر الگوهای مشابه پرداخته شده است.

\newpage
\section{شرح سیستم برنامه‌ریزی منابع سازمانی}
سیستم برنامه‌ریزی برای منابع سازمانی، یکی از انواع سیستم‌های کسب و کار است که مدیریت منابع سازمانی شامل منابع انسانی، منابع اطلاعاتی، منابع مالی (نقدی و غیرنقدی) و منابع فیزیکی را تسهیل می‌کند. سیستم برنامه‌ریزی پیش رو برای استفاده در سازمانی ایجاد شده است که در حوزه‌ی ایجاد سیستم‌های نرم‌افزاری فعالیت می‌کند. \par
در این سازمان هر پروژه شامل تعدادی سیستم و هر سیستم خود شامل تعدادی ماژول است. در هر پروژه تیمی متشکّل از افرادی از واحدهای چهارگانه‌ی مهندسی نیازمندی‌ها، تحلیل، طرّاحی، و پیاده‌سازی و نگه‌داری سازمان فعالیّت می‌کنند. همچنین هر فرد در این سازمان با یکی از مراتب شغلی کارمند معمولی، مدیر میانی و مدیر سطح بالا فعالیت می‌کند. سطح دسترسی و میزان تعامل هر فرد با سیستم برنامه‌ریزی در هر پروژه به طور جداگانه و توسط فردی با مرتبه‌ی شغلی بالاتر تعیین می‌گردد. \par
سیستم برنامه‌ریزی قابلیت ثبت فرایند ایجاد و نگه‌داری را پشتیبانی می‌کند که امکان ثبت اعضا، زمان صرف شده، و منابع استفاده شده در روند ثبت و تغییر ماژول‌ها را در اختیار کاربران می‌گذارد. این سیستم همچنین امکان ثبت منابع موجود، ثبت نیازمندی‌های کنونی و ثبت اندازه‌ی پروژه‌های نرم‌افزاری را فراهم می‌کند. به وسیله‌ی همین اطلاعات است که امکان گزارش‌گیری از سیستم برنامه‌ریزی با هدف دریافت گزارش منابع موجود و مورد نیاز و دریافت گزارش جریان چرخشی مصرف فراهم می‌شود. همچنین این سیستم با کمک همین اطلاعات در پیش‌بینی برای تخمین منابع و یافتن نیازمندی‌های ضروری به کاربران کمک می‌کند. \par

