
\chapter{فهرست اولویت‌بندی‌شده نیازمندی‌ها}
برای اولویت‌بندی نیازمندی‌ها از یک رویکرد دوگامی استفاده کرده‌ایم. در گام اول نیازمندی‌ها بر اساس قواعد MuSCoW و در واقع ارزش آن‌ها برای کاربر اولویت‌بندی شده‌اند. در گام دوم، نیازمندی‌ها در هر دسته، بر مبنای میزان ریسکی که می‌توانند برای پروژه داشته باشند، اولویت‌بندی می‌شوند. بر اساس تحلیل مقدماتی نیازمندی‌ها، همه‌ی نیازمندی‌هایی که به دست آمده‌اند، Must have هستند. در نتیجه در هر زیرسیستم، نیازمندی‌ها بر اساس میزان ریسکشان اولویت‌بندی شده‌اند؛ از اولویت بیشتر به کمتر. \\

\section{زیرسیستم تولید و نگه‌داری}
\paragraph{توجیه اولویت‌بندی}
اضافه کردن منابع به پروژه به دلیل اینکه انواع مختلفی دارند و هر کدام شامل صفات گوناگونی هستند ممکن است با مشکل مواجه شود. یکی از مشکلات احتمالی، این است که صفاتی در مورد برخی از منابع مد نظر مشتری باشد اما به روشنی بیان نشده باشد یا با روشن‌شدن آن‌ها در ادامه، محدوده‌ی پروژه بزرگتر شود. در نتیجه این نیازمندی بیشتر اولویت را دارا خواهد بود. در سطح بعدی اولویت، پنج نیازمندی مرتبط با ثبت نوع تغییر قرار خواهند گرفت؛ چرا که انواع تغییراتی که ممکن است مد نظر مشتری باشد به روشنی بیان نشده است یا ممکن است در آینده تغییرات زیادی کند. چهار مورد آخر هم در اولویت‌های آخر قرار می‌گیرند؛ چرا که در تحلیل مقدماتی، به خوبی روشن شده‌اند.
\begin{enumerate}
	\item اضافه کردن منابع به پروژه
	\item تغییر ماژول یک سیستم
	\item ثبت تغییر دهنده‌ی ماژول
	\item ثبت نوع تغییر ماژول
	\item ثبت میزان زمانی که تغییر ماژول برده
	\item ثبت منابع استفاده شده برای تغییر ماژول
	\item اضافه کردن پروژه
	\item اضافه کردن سیستم به پروژه
	\item اضافه کردن ماژول به سیستم
	\item ثبت ایجادکننده یا ایجادکنندگان ماژول
	\item ثبت زمان صرف شده برای ایجاد ماژول
	\item ثبت انواع منابع صرف شده برای ایجاد ماژول
	\item حذف کردن پروژه
	\item حذف سیستم از پروژه
	\item حذف ماژول از سیستم
	\item مشاهده مشخصات منبع
	\item مشاهده حزئیات پروژه
	\item  مشاهده فهرست پروژه ها
\end{enumerate}
\section{زیرسیستم توزیع}
\paragraph{توجیه اولویت‌بندی}
ثبت اطلاعات مربوط به استفاده شدن یک منبع، به صورت خیلی کلی مطرح شده است و ریسک درک نادرست آن را تهدید می‌کند. در نتیجه از بیشترین اولویت برخوردار است. دیگر نیازمندی‌ها، در یک سطح از اولویت قرار خواهند گرفت.
\begin{enumerate}
	\item ثبت اطلاعات مربوط به استفاده شدن یک منبع (چه زمانی، کدام قسمت سازمان، کدام بخش از فرایند ایجاد یک پروژه نرم‌افزاری)
	\item اضافه کردن منبع به واحد سازمان
	\item تغییر مشخصات یک منبع
	\item ثبت نیازمندی‌های کنونی واحدهای مختلف
	\item ثبت اینکه نیازمندی خاصی در یک واحد رفع شده یا نه
	\item ثبت زمان رفع شدن یک نیازمندی در یک واحد سازمان
	\item افزودن واحد به سازمان
	\item ثبت تعداد نیروی انسانی ایجاد کننده برای هر پروژه
	\item ثبت تعداد نیروی انسانی استفاده کننده برای هر پروژه
	\item ثبت تعداد ماژول‌های یک پروژه
	\item ثبت نام تکنولوژی مورد استفاده برای پروژه
	\item ثبت هدف بکارگیری یک تکنولوژی
	\item حذف منبع از واحد سازمان
	\item حذف واحد از سازمان
	\item مشاهده فهرست نیازمندی های یک واحد
	\item مشاهده فهرست منابع یک واحد
\end{enumerate}

\section{زیرسیستم گزارش‌گیری}
\paragraph{توجیه اولویت‌بندی}
نیازمندی‌های این زیرسیستم، همگی از نوع must have هستند. اما در صورت برآورده‌شدن نیازمندی‌هایی که در چند زیرسیستم قبل به آن‌ها اشاره شد، از اطلاعات موجود در آن‌ها می‌توان برای گزارش‌گیری استفاده کرد. در نتیجه پنج نیازمندی زیر همگی در یک سطح قرار می‌گیرند.
\begin{enumerate}
	\item گزارش منابع موجود (شامل میزان موجود از هر منبع و قسمتی از سازمان که از آن استفاده می‌کند)
	\item گزارش اینکه هر منبع در هر زمان از یک بازه زمانی معلوم در اختیار چه پروژه یا واحدی بوده است
	\item برای هر پروژه چه منابعی مورد نیاز است (حال حاضر)
	\item برای هر پروژه چه منابعی مورد نیاز بوده است (قبلا)
	\item دو گزارش بالا برای همه پروژه‌ها 
\end{enumerate}

\section{زیرسیستم پیش‌بینی}
\paragraph{توجیه اولویت‌بندی}
اعمال فیلترهای چندگانه، از این جهت اهمیت دارد که قرار است مبنایی برای کمک به مدیران پروژه در انجام وظایفشان شود. همچنین، در این مورد ابهام‌هایی در ابتدای تحلیل مقدماتی وجود داشت. در نتیجه اعمال فیلترها بیشترین اولویت را خواهند داشت. در مرتبه‌ی بعدی امکان انجام انواع جستجو مطرح می‌گردد. دلیل قرار گرفتن این موارد پس از فیلترها، این است که با وجود اهمیت، ابهامی در مورد آن‌ها به وجود نیامده است. نمایش نتایج حاصل از جستجوها در آخر قرار می‌گیرد؛ چرا که طبق تجربیات پیشین، پیچیدگی‌های غیرقابل حلی ندارد. در نتیجه اولویت‌بندی به صورت زیر انجام خواهد گرفت. (موارد ۳ و ۴ در یک سطح قرار می‌گیرند.)
\begin{enumerate}
	\item امکان اعمال فیلترهای چندگانه باید وجود داشته باشد.
	\item امکان فیلتر کردن بر مبنای یک ویژگی هم وجود داشته باشد.
	\item جستجو بر اساس سایز پروژه (مقدار دقیق) در بین پروژه‌هایی که تا به حال انجام شده
	\item جستجو بر اساس تکنولوژی بین پروژه‌هایی که تا به حال انجام شده
	\item جستجو بر اساس یک یا چند منبع - بر اساس مقدار دقیق (و نه بازه)
	
	\item نمایش نتایج حاصل از انواع  جستجوها، باید شامل تمام منابع استفاده شده در آن پروژه‌ها به همراه زمان تامین شدن آن منبع باشد.
\end{enumerate}

\section{زیرسیستم کاربری}
\paragraph{توجیه اولویت‌بندی}
در سیستم برنامه‌ریزی، سطوح دسترسی کاربران از اهمیت بسزایی برخوردار است. با توجه به تجربیات پیشین، مساله‌ی تعیین دسترسی می‌تواند پیچیدگی‌هایی به همراه داشته باشد یا در تحلیل مقدماتی به صورت  شفاف مطرح نشده باشد. در نتیجه از بیشترین اولویت برخوردار است. در صورتی که این موضوع حل شود، تغییر این سطوج راحت‌تر انجام خواهد گرفت و در نتیجه از اولویت پایین‌تری برخوردار خواهد بود. از طرفی ورود و خروج از سیستم، تعریف روشنی دارند و کمترین اولویت را در این زیرسیستم به خود اختصاص می‌دهند. در نتیجه اولویت‌بندی از قرار زیر خواهد بود:
\begin{enumerate}
	\item تعیین سطح دسترسی کاربر دیگر
	\item تغییر سطح دسترسی کاربر دیگر
	\item ثبت نام در سیستم
	\item تایید ثبت نام
	\item ورود به سیستم
	\item خروج از سیستم			
\end{enumerate}

\section{زیرسیستم پشتیبانی}
\paragraph{توجیه اولویت‌بندی}
تهیه نسخه پشتیبان نیاز به امکانات سخت افزاری با قابلیت بالایی دارد که امکان دارد موجود نباشد یا به دلایل دیگر امکان تهیه آن وجود نداشته باشد.
\begin{enumerate}
	
	\item تهیه نسخه پشتیبان
\end{enumerate}

\section{Requirements Significant Architecturally}
({\color{red} بهنگام‌سازی شد.}	)
\paragraph{توضیح}
نیازمندی‌های با اولویت بالا و ریسک بالا را می توان در این دسته از نیازمندی‌ها در نظر گرفت. نیازمندی‌های رایجی که در این دسته قرار می گیرند، مربوط به تکنولوژی‌های مورد استفاده در پروژه است. برخی از نیازمندی‌های این دسته عبارتند از:
\begin{enumerate}
	\item استفاده از زیان برنامه‌نویسی جاوا
	\item استفاده از JDBC برای ارتباط با پایگاه داده
	\item استفاده از پایگاه‌داده MySQL
	\item استفاده از Swing و AWT برای پیاده‌سازی واسط کاربری
\end{enumerate}

\section{توضیحاتی راجع به Baseline Architectural Executable}
نیازمندی‌هایی که در تکرار اول به عنوان 
\lr{Architecturally Significant}
در نظر گرفتیم مربوط به تکنولوژی پیاده‌سازی پروژه است. بدین ترتیب هیچ یک از موارد کاربردی که در آن مرحله تعریف شده جزو
\lr{Architecturally Significant Requirement}
 نیست. همچنین، تاکنون بازخوردی که باعث افزوده شدن نیازمندی‌های جدید شود دریافت نشده است. در نتیجه طی صحبتی که با دستیار محترم آموزشی سرکار خانم دهقانی داشتیم، در این مرحله
 \lr{Executable Architectural Baseline} ای
 که در مرحله‌ی قبل پیاده‌سازی شده تغییری نخواهد کرد.